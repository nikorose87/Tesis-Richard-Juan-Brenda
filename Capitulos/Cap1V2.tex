\cleardoublepage %poner esta linea al inicio de cada capitulo
% La presentación se hará a través de un video de una extensión de cinco minutos

%2. El contenido de la presentación debe incluir los siguientes elementos

%• Título del proyecto,

%• Justificación,

%• Planteamiento del problema y pregunta de investigación,
%• Objetivos: general y específicos,
%• Alcances y limitaciones,
%• Marco Conceptual/Antecedentes/Estado del Arte,
%• Metodología,
%• Cronograma de Actividades,
%• Presupuesto.
%3. El video o enlace de la presentación debe ser enviado al profesor Horacio Coral al correo hcoral@usbbog.edu.co antes de las 5:00 p.m. del día 08 de junio de 2021.


\chapter{Introducción}\label{sec:capitulo1}

\section{Estado del arte}
\subsection{Proceso del café}

El café Colombiano es uno de los más suaves y de mejor calidad a nivel mundial por las variedades cultivadas, las condiciones ambientales, la recolección manual selectiva \citep{Calidadd}. Los granos de cafe pasan por un proceso extenso antes de tostarlos y prepararlos en una bebida, es necesario conocer cuál es el proceso por el cual pasan, empezamos desde un buen cultivo, en el que es indispensable tener una siembra acertada de café ya que, si éstos son óptimos, los rendimientos en la producción serán buenos, contrario a lo que ocurre cuando el crecimiento del cultivo es deficiente \citep{Crecimiento}.\medskip 

La cosecha de café como se realiza en Colombia, es una de las etapas más importantes del cultivo, pues en ella se recoge el producto del esfuerzo que año tras año llevan a cabo los caficultores colombianos. Pero no sólo por esta razón es importante la cosecha, también se conoce que, aunque es variable, esta etapa puede representar el 40\% de los costos de producción del café y que además es una labor absolutamente dependiente de la mano de obra para su ejecución \citep{Cosecha}.\medskip

En la parte de post-cosecha del café tenemos un proceso húmedo o seco que deja al grano cubierto sólo con una cáscara llamada pergamino, que una vez retirada produce un grano limpio y listo para su comercialización o para la continuación del proceso, nombrado café verde. Cada fruto de café está compuesto por diferentes capas: la piel externa, llamada exocarpio; la pulpa o mesocarpio; el mucílago, una capa viscosa debajo de la pulpa que aporta el dulzor del café; el pergamino o endocarpio y el tegumento, una película plateada que recubre las \citep{Beneficio}.\medskip

El tostado es uno de los pasos más importantes para finalizar el proceso, es responsable de los cambios químicos y físicos que sufrirá el grano. Se sabe que el color es uno de los parámetros más usados para evaluar la calidad del producto final, este es quien establece el nivel de tostado del café \citep{Tostado}. A su vez la temperatura \citep{TuesteTemp} de tueste juega un papel fundamental para tener una optima calidad, la cual podremos ver en la siguiente sección. Teniendo en claro estos procesos nos enfocaremos en la cosecha, post-cosecha y tueste logrando mejorar la calidad del cafe.


\subsection{Características de calidad}

La calidad del café es un rasgo importante al momento de exportar el café, entre los componentes que determinan esta calidad permanecen las condiciones de clima, suelo, funcionamiento del cultivo, procedimiento postcosecha y la manera de preparación de la bebida. Los males en el grano; como la infección de insectos como la broca del café, ocasionan deterioro en la calidad física, sanitaria y organoléptica del café, provocando considerables pérdidas económicas y de calidad, aun cuando, es de anotar que la mayor parte de las deficiencias se originan en el proceso de postcosecha \citep{puerta1999influencia}\medskip

La recolección de frutos maduros y sanos son los primeros requisitos para obtener café de buena calidad \citep{puerta2016calidad} y por lo consiguiente granos no defectuosos. Los granos defectuosos representan en torno al 15\% al 20\% de la producción de café en peso. Todos los años se crean alrededor de 1,2 a 1,5 millones de toneladas en la industria del café y son rechazadas en el mercado mundial, debido a que dichos granos defectuosos generan sabores indeseables en la bebida una vez que se tuestan junto con granos clasificados \citep{ramalakshmi2007physicochemical}. Durante el cultivo y el proceso de beneficio del café pueden generarse unos veinticinco defectos, el 80\% de los cuales son originados por un inadecuado beneficio y almacenamiento del grano \cite{puerta2001como}. Estos defectos son clasificados por la Federación Nacional de Cafeteros de Colombia en dos grupos como se puede observar en la Tabla \ref{tab:grupos_defectos} , siendo el primer grupo los defectos que mas afectan en la calidad del café \cite{resolucion}. \medskip

\begin{table}[h]
    \begin{center}
        \caption{Grupos de defectos en el café}
        \begin{tabular}{ | m{2cm} | m{10 cm} | }
        \hline 
        \multicolumn{2}{|c|}{Defectos en los granos de café} \\ \hline
        Grupo 1 &  Granos negros llenos, parciales o secos, vinagres enteros o parciales, reposados amarillos o carmelitas y ámbar o mantequilla. \\ \hline
        Grupo 2 &  Grano flojo, decolorado ( veteado y blanqueado ),\newline cardenillo, mordido o cortado, picado por insectos, sobre secado o quemado, partido, malformado o deformado, inmaduro, aplastado, flotador o balsudo, averanado o arrugado.  \\ \hline
        \end{tabular}
        \label{tab:grupos_defectos}
    \end{center}
\end{table}

La colimetría del café ha de ser un factor desicivo al momento de clasificar la calidad del café debido a que esta determina el término de madurez del grano o fruto el cual varia entre los colores verde (primeras etapas), amarillo o naranja (proceso de maduración), rojo (maduro) y violeta (sobremaduro) \citep{herrera2011colorimetria}. Para una buena calidad de color se debe tener en cuenta que el café debe tener una apariencia uniforme \citep{resolucion}. \medskip

Otro de los factores importantes al momento de la clasificación de calidad es el porcentaje de humedad. Si los frijoles están demasiado húmedos (por encima del 12,5 \% de humedad), se moldearán fácilmente durante el almacenamiento y el café en almendra se deteriora dando asi un aspecto decolorado \cite{puerta2001como}. Si los frijoles están demasiado secos (por debajo del 8 \% de humedad), perderán sabor \citep{leroy2006genetics}. Debido a las mezclas de café con contenidos diferentes de humedad y por el rehumedecimiento o interrupción del secado antes que los granos almacenen el 12\% de humedad, se generan defectos en estos, los cueles son, grano sucio, mohoso, decolorado, cristalizado y se favorecen defectos como la producción de micotoxinas \cite{puerta2001como}. \medskip

Otros requisitos para una buena cálidad de café para exportación dada por la Federación Nacional de Cafeteros de Colombia son el olor, la prueba de taza y la infestación, la cual indica que el café debe estar libre de todo insecto vivo, tal como la Broca, dado que estos logran un gran deterioro en el grano del café \citep{resolucion}.

\subsection{Procesamiento de imágenes}

\subsubsection{Antecedentes}
\begin{itemize}
    \item En la universidad distrital francisco José de Caldas en Colombia diseñaron un modelo de aprendizaje automático que predice la calidad del café, hicieron uso de algoritmos como árbol de decisión, vecinos cercanos (KNN) y redes neuronales, se tomaron fotos de grupos que contaban con 200 a 400 gr de café, las características evaluadas se tienen en cuenta con el comité Nacional de Cafeteros, el resultado obtenido fue de un 83\%  de precisión en los algoritmos de vecinos cercanos y red neuronales, la diferencia que se aprecia en estos dos algoritmos es que con los vecinos cercanos se realiza una recolección de datos para un puntaje alto de café y con el de redes neuronales se obtienen datos para determinar un puntaje bajo del café.\citep{suarez2019modelo}.

     \item 	En la universidad de Piura se diseño un sistema de procesamiento de imágenes haciendo uso de las librerías de OPENCV aplicadas en una RASPBERRY PI para la clasificación del café, para realizar el proceso de entrenamiento del sistema se utilizo un total de 600 fotos (300 frontales y 300 de perfil) , las características principales  rescatar son el Tamaño, color y estado de fermentación (bien fermentado, parcialmente fermentado y mal fermentado) el sistema determina las malformaciones de las muestras tales como granos pegados o dañados usando los datos obtenidos con el comando regionprops seleccionando el parámetro de puntos convexos, también se identifica los elementos que no sean granos de café esto se realizan teniendo en cuenta las características del color externo del grano haciendo usos de los modelos de color HSV,  los resultados obtenidos fueron de un 89 \% de precisión teniendo en cuenta el análisis en 25 de las 600 muestras, con respecto al tamaño se tuvo un error de 0.06 cm.\citep{viera2017procesamiento}.

\end{itemize}

\section{Planteamiento y Justificación del Problema}

\subsection{Justificación}

La clasificación del café es una labor realizada por catadores profesionales que están especializados en poder determinar los atributos del café, a el bajo número de personas capacitados para dicha labor y el número tan grande de hectáreas a evaluar en un tiempo limitado es uno de los problemas en la recolección y clasificación del café causando así un paradigma en cuanto a cantidad de café producido con respecto a la baja calidad con la que sale al mercado. \medskip

El sistema en el que esta organizado la cadena de producción de café en Colombia dificulta la selección y clasificación de la calidad del café, esto se debe a factores tales como la época, la región y los tipos de terrenos en los que se cosecha aumenta el margen de error en la selección de la calidad del café. \medskip
 
Teniendo en cuenta dichas problemáticas se requiere de un método o análisis numérico que determine las propiedades fisicoquímicas al momento de analizar el café en los procesos de cosecha, post-cosecha y tostado asegurando así un producto final de mayor calidad. \medskip

Las técnicas de aprendizaje automático garantizan la capacidad de las máquinas de reconocer patrones, características esto se logra mediante el uso de datos o imágenes que permitan realizar un proceso de entrenamiento que permite relacionar datos de entrada con datos de salida, esto con el fin de poder garantizar la calidad del café sin importar las condiciones de cosecha.

%1 o dos párrafos en donde sinteticen las problemáticas del estado del arte.
\begin{itemize}
    \item Cosecha: Tamaño y color %Buscar vacios en está área
    \item post-cosecha: Grano limpio %Buscar vacios en está área
    \item Tueste: Temperatura? %Explicar en cual item de calidad nos vamos a enfocar
    \item cumplimiento de equivalencias por las asociaciones.
    \item Humedad ? 
\end{itemize}

\subsection{Pregunta de investigación}

%NO SE CREA, SE IMPLEMENTA
¿Qué sistema de clasificación automática permitirá identificar los defectos del café en el proceso de 
cosecha, post-cosecha y tostado, obteniendo una mejor calidad en el producto final?
 
 
\subsection{Hipotesis}

Un modelo de aprendizaje automático y visión artificial, cuyas variables se determinarán por las características de los granos de café, brindará un mejoramiento de los procesos de selección de grano y así, proveer un producto de la calidad esperada. 

 
\section{Objetivos}


\subsection{Objetivo General}
%Tiene solo un verbo rector

Desarrollar un sistema automático de clasificación \--- acorde a las características físicas \--- de granos de café mediante técnicas de visión artificial para los procesos de cosecha, postcosecha y tostado.\medskip


\subsection{Objetivos Específicos}
\begin{itemize}
    \item Recopilar el dataset acorde a las especificaciones técnicas para selección de grano de café.

    \item Determinar el algoritmo de aprendizaje automático que clasifique los granos de café que no cumplan con las características de calidad.
    
    \item Validar la efectividad del modelo de aprendizaje automático mediante una prueba piloto con un prototipo propuesto.
\end{itemize}
	
\section{Alcances y Limitaciones}

%Parrafo de alacances y limitaciones

\subsection{Alcances}

Con el proyecto se busca determinar la viabilidad del uso de procesamiento de imágenes e IA para clasificar la calidad del café. %Mencionar hasta donde se va a llegar

\begin{itemize}

\item obtener un numero adecuado de muestras de café con el fin de poder estudiar todos los tipos de café que se cosechan en Colombia.

\item Diseñar un sistema controlado que garantice obtener todas las características del café en una imagen.

\item	Seleccionar un algoritmo capaz de analizar y capturar las características del café haciendo uso de entrenamientos que determinara todas las posibles variaciones del café.

\item Determinar la calidad del café en los procesos de cosecha, post- cosecha y tostado del café ofreciendo así un producto de mayor calidad en el mercado.


\end{itemize}

\subsection{Limitaciones}

Las muestras con las que se realizará el proceso de entrenamiento del sistema fueron seleccionadas de una regional especifica de Colombia, esto permite realizar una evaluación inicial del sistema en la región cundinamarqués. No obstante, el desempeño del sistema se puede ver perjudicado si se analizan muestras de café de regiones diferentes a la inicial, esto generará que se tenga que realizar un proceso de re-entrenamiento para poder abarcar una mayor cantidad de tipos de café.



\section{Metodología}

El método a usar es el deductivo, se parte desde un análisis general de las imágenes obtenidas de los frutos de café, estas imágenes contienen información del estado del fruto, llegando a un análisis particular etiquetando cada objeto presente en la imagen: café bueno o café malo.\medskip

A continuación se detallan los pasos metodológicos para el proceso de selección de frutos de café mediante visión artificial y IA. 

%Critical Path method
%SCRUM
%Agile
%Waterfall <- engineering


\section{Cronograma de actividades}

\newpage

\section{Presupuesto}

\begin{center}

\begin{tabular}{| c | c | c |}
\hline
Descripción  &  & Valor Unitario COP \\ \hline
Personal &  & 4 \\
Equipos &  & 10 \\
Mantenimiento &  & 10 \\
Gastos de viaje &  & 4 \\
Publicaciones &  & 10 \\
Total &  & 3 \\ \hline
\end{tabular}

\end{center}